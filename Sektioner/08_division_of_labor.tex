
\section{Arbejdsfordeling}

Arbejdsfordelingen har fungeret horisontalt. Det vil sige, at der ikke har været ansvarsområder under projektets udvikling, men alle medlemmer har haft mulighed for at arbejde på kryds og tværs af emner i projektet. Dog med den undtagelse af selve opsætningen af selve projektet, som skete før bachelor-gruppen gik igang med selve udviklingen. Men til alle de forskellige funktionaliteter  har opgaverne været funktions-bestemt - altså opdelt efter de forskellige user stories.

Dette har betydet at alle medlemmer er endt med at arbejde både på Converge-SPA og Converge-cluster. På den ene eller anden måde. Både i form af dokumentation, eller implementering eller design.

Derfor er arbejdet ikke som sådan blevet fordelt, men har i stedet foregået således, at der til hvert sprintmøde er blevet oprettet alle de arbejdsopgaver på baggrund af krav, samt estimeret en tid hvor lang tid disse skulle tage. Opgaverne blev derefter trukket ind på Scrum boardet, hvorefter det har stået frit for gruppemedlemmerne hvilke arbejdsopgaver de ville arbejde med i det pågældende sprint.

Det har også været forventet, at hvert gruppemedlem arbejdede det der svarer til 25-30 timer om ugen på projektet, og derfor har hvert gruppemedlem også ansvaret for at nå de respektive timer, når de tog opgaver fra Scrum-Boardet. Hvis en arbejdsopgave tog længere eller kortere tid, blev Scrum-Boardet justeret løbende. 

Efter et par sprints var det muligt at estimere ca. hvor meget gruppen kunne nå på en sprintlængde, og der har løbende været brugt til estimering af opgaver samt fastlægning af deadlines.

Arbejdsfordelingen har som princip fungeret godt i gruppen, men da de forskellige medlemmer kom med deres individuelle styrker, har der været nogle opgaver som blev taget oftere af den samme person, hvilket har gjort projektet en smule polariseret. Men det mener gruppen ikke har været en negativ ting, da på den korte udviklingstid, har gruppen haft mulighed for at lave et så ambitiøst og komplet produkt som muligt.