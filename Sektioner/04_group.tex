\chapter{Gruppedannelse}

En del af projektforløbet omhandlede gruppedannelsen, bachelor-gruppen har selv fundet sammen, fordi de to medlemmer hver især kom med deres styrker. De to medlemmer har arbejdet sammen på mange af de tidligere semestre og havde kenskab til hinanden inden udviklingen. Dette har også gjort at der var et fælles ambitionsniveau for gruppen.

Netop det fælles ambitionsniveau var vigtig for alle i gruppen, og det blev derfor både fastlagt før gruppedannelsen, men også meget tidligt efter gruppedannelsen, for at have en klar forventningsafstemning blandt gruppens medlemmer. Derfor kunne arbejdsbyrden estimeres for gruppen og mindske eventuelle konflikter. Diversitet har også været vigtigt, eftersom gruppens medlemmer kom med hver deres styrker. Sameer med Frontend udvikling og Kasper med DevOps og backend kompetancer.