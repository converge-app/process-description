\chapter{Udviklingsforløb}

I løbet af udviklingen af dette projekt er der brugt Scrum. Scrum er en agil arbejds- og planlægnings metode, der går ud på at man bestemmer, hvordan arbejdet i de kommende sprints skal foregå. Scrums styrker ligger blandt andet i planlægning og arbejdsoversigt, hvilket er med til at skabe en hurtig leveringsproces af et stykke arbejde. Scrum er brugt i løbet af selve udviklingsprocessen af produktet, og har været inddelt i 1-2 ugers sprint, hvor længden af sprintet har været sat efter behov. Inden starten af et sprint har der været sprintmøder, hvor der blev planlagt en passende mængde arbejde til det kommende sprint. Dermed fik gruppens medlemmer en passende mængde arbejdsopgaver.

Efter sprintet var startet, ville gruppens medlemmer vælge passende mængde opgaver, der matchede deres ønsker. Dette har resulteret i en løs, men grundig arbejdsform. I slutningen af sprintet var der planlagt sprint retrospective og review. Retrospectives ser tilbage og ser på hvordan sprintet var, og hvordan arbejdsmetoden kunne forbedres. Under sprint review ses tilbage på arbejdet udført i det forgangne sprint. Her ses på resultatet af arbejdet og de nyeste tilføjelser til produktet. Sprint reviewet blev ofte afsluttet med en demo af nuværende tilstand af produktet. Efter nogle af de store sprints, hvor der har været væsentlig udvikling på produktet, blev der oprettet en udgivelse i forhold til de repositories gruppen brugte til at opbevare dokumentation og kode. Dette resulterede i at gruppen havde versionering på væsentlige dokumenter og kode.

Det er væsentligt at nævne Git og GitHub. Git har været brugt som source control manager, og har været et under-læggende værktøj, alle gruppens medlemmer har brugt til at dele kode. Før at Git kan bruges til at dele kode med andre, kræver det at der bruges en Repository Manager, hvor valget faldt på GitHub. GitHub har været væsentlig fordi det har tilladt gruppen at dele kode på internettet, og det har også gjort at gruppen kunne bruge GitHub work- flow. GitHub workflow går ud på at gruppen kan arbejde isoleret, men at man kan modtage kommentarer til det man har lavet, alt sammen før man fletter sin kode sammen med det der skal udgives. Det har især været væsentligt, eftersom gruppen har brugt continuous integration til at teste den kode, der blev udgivet ved hjælp af GitHub. Dette har været helt essentielt, da gruppen nu kunne bruge de tests fremstillet til koden, til at kontrollere om det der blev fremstillet overholdte de prædefinerede standarder, defineret ved hjælp af unit-, integration- og systemtests. Det har også gjort at gruppen altid havde en funktionel applikation, hvilket kunne fremlægges til sprint review møder.

