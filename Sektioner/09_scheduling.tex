
\section{Planlægning}

Til planlægningen af projektforløbet har det tidsmæssige aspekt spillet en stor rolle. Der er som udviklingsmetode benyttet Scrum, hvor der er benyttet henholdsvis 1- og 2-ugers sprints.

Det meste af projektet blev udført med 2-ugers sprints, da teamet havde en klar opdeling af krav og prioriteter for designet af produktet. Derudover blev 1-ugers sprint brugt til sidst under rapport skrivningen. 2-ugers sprints blev valgt på den baggrund at en funktion skulle kunne designes, implementeres og reviewes inden sprintet var ovre, teamet mente ikke det kunne nås på en uge.

Til planlægning blev der brugt 2 cerimonier fra Scrum, refinement og planning. Refinement har været at klargøre backloggen med opgaver, så de er estimerede og veldefinerede. Planning har været analyse af hvordan det pågældende sprint skulle se ud og hvad der skulle prioriteres og laves. Her har teamet diskuteret hvad der skulle laves, og hvorfor, samt hvem skulle lave hvad.

Til planlægningen er ZenHub og GitHub brugt som de primære værktøjer, samt TeamGantt til at se hvornår de respektive deadline lå, samt for den generelle tidsplan.